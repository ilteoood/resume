\documentclass[10pt,a4paper]{altacv}

\geometry{left=1cm,right=9cm,marginparwidth=6.8cm,marginparsep=1.2cm,top=1.2cm,bottom=1cm}
\usepackage[utf8]{inputenc}
\usepackage[T1]{fontenc}
\usepackage[default]{lato}
\usepackage[pdfborder={0 0 0}]{hyperref}

\definecolor{VividPurple}{HTML}{AA1645}
\definecolor{SlateGrey}{HTML}{2E2E2E}
\definecolor{LightGrey}{HTML}{666666}
\colorlet{heading}{VividPurple}
\colorlet{accent}{VividPurple}
\colorlet{emphasis}{SlateGrey}
\colorlet{body}{LightGrey}

\renewcommand{\itemmarker}{{\small\textbullet}}
\renewcommand{\ratingmarker}{\faCircle}

\addbibresource{publications.bib}

\begin{document}
\name{Matteo Pietro Dazzi}
\tagline{Senior Software Engineer}
\photo{2.5cm}{me}

\personalinfo{
% Add your own with \printinfo{symbol}{detail}
  \email{\href{mailto:matteopietro.dazzi@gmail.com}{matteopietro.dazzi@gmail.com}}
  \phone{+39 (370)-344-6446}
  % \mailaddress{}
  \location{Rho, MI 20017, Italy}
  %\homepage{\href{http://ilteoood.xyz}{ilteoood.xyz}}
  %\twitter{\href{https://twitter.com/ilteoood}{@ilteoood}}
 \linkedin{\href{https://www.linkedin.com/in/iLTeoooD}{Matteo Pietro Dazzi}}
 \github{\href{https://github.com/iLTeoooD}{iLTeoooD}}
 \medium{\href{https://ilteoood.medium.com/}{@ilteoood}}
 \homepage{\href{https://ilteoood.xyz}{https://ilteoood.xyz}}
}

\begin{adjustwidth}{}{-8cm}
\makecvheader
\end{adjustwidth}

\cvsection[sidebar_first_page]{Work}

\cvevent{Senior Software Developer}
{\href{https://www.nearform.com/}{NearForm}}
{September 2022 -- Now}{Remote, Italy}
Development and maintenance of NearForm's open source projects.
\newline\newline
Technologies involved:
\begin{itemize}
	\item JavaScript (React \& Node.js);
\end{itemize}

\divider

\cvevent{Senior Technical Leader}
{\href{https://mia-platform.eu/}{Mia-Platform}}
{February 2022 -- September 2022}{Milan, Italy}
Maintainer of the \textbf{micro-lc} organization and technical leader of the \textbf{Mia-Platform Console} product.
\newline\newline
Technologies involved:
\begin{itemize}
	\item JavaScript (React \& Node.js);
	\item Go;
	\item MongoDB;
	\item Docker;
	\item Kubernetes
\end{itemize}

The work method follows the \textbf{Scrum Principles}.
\divider

\cvevent{Expert Full Stack Developer}
{\href{https://mia-platform.eu/}{Mia-Platform}}
{March 2021 -- February 2022}{Milan, Italy}
Maintainer of the \textbf{micro-lc} organization and technical contact for the following products:
\begin{itemize}
	\item Mia-Platform Console;
	\item Developer Portal;
	\item Swagger Aggregator;
	\item Back-office
\end{itemize}
\leavevmode \newline
Technologies involved:
\begin{itemize}
	\item JavaScript (React \& Node.js);
	\item Go;
	\item MongoDB;
	\item Docker;
	\item Kubernetes
\end{itemize}

The work method follows the \textbf{Scrum Principles}.
\divider

\newpage

{\marginpar{\raggedright\cvsection{Experience}
\vspace{2mm}
\wheelchart{1.1cm}{0.4cm}{
  40/4em/accent!100/Front End Engineer,
  30/6em/accent!60/Back End Engineer,
  20/6em/accent!30/Automation,
  10/6em/accent!10/DevOps \& Cloud
}
\vspace{1mm}

\cvsection{Side Projects}
\vspace{1mm}
\cvachievement
{\faBolt}
{\href{https://github.com/orchy-mfe}{orchy}}
{The Micro Frontend orchestrator}
\vspace{2mm}
\divider
\cvachievement
{\faFlag}
{\href{https://github.com/ilteoood/flutter_i18n}{flutter\_i18n}}
{I18n made easy, for Flutter!}
\vspace{2mm}
\divider
\cvachievement
{\faGithub}
{\href{https://github.com/ilteoood/docker_buildx}{docker\_buildx}}
{GitHub Action to build and publish Docker images using Docker Buildx.}
\divider
\cvachievement
{\faConnectdevelop}
{\href{https://github.com/ilteoood/docker-surfshark}{docker-surfshark}}
{Docker container with OpenVPN client preconfigured for SurfShark VPN.}
}}
\cvevent{Full Stack Software Engineer}
{\href{https://studioform.org/}{GFT Italia S.r.l - for Studioform}}
{October 2020 -- December 2020}{Remote working}
In this project I have joined 2 different teams, each ones with at least 4 developers.
\newline
The work method follows the \textbf{Scrum Principles}.
\newline\newline
Dedalus Team, \textbf{Front-End}:
\begin{itemize}
	\item \textbf{Reorganization of the structure} of the \textbf{Angular 10} app;
	\item Realization of core functionality of a text editor, based on \textbf{\href{https://prosemirror.net/}{Prosemirror}};
	\item Creation of the \textbf{TypeScript} types;
	\item Definition and execution of the \textbf{linting rules};
	\item Production of the \textbf{Angular templates};
	\item Training colleagues on \textbf{Angular} ecosystem.
\end{itemize}
\leavevmode \newline
Icarus Team, \textbf{Full Stack}:
\begin{itemize}
	\item Creation and publication of custom \textbf{Docker} images on \textbf{Docker Hub};
	\item Deployment and versioning of \textbf{Docker Compose} files;
	\item \textbf{Docker Compose} stacks management with \textbf{Portainer};
	\item Deployment and management of a \textbf{RabbitMQ cluster};
	\item Realization of a new microservice and \textit{Rest APIs} in \textbf{Java 8} and \textbf{Spring Boot};
	They have been configured to work with \textbf{Eureka} and they have been tested with \textbf{JUnit 5};
	\item Evolution of the \textbf{Angular 8} application.
\end{itemize}
\divider

\cvevent{Lead Front End Engineer}
{GFT Italia S.r.l - for Pegasus Project}
{August 2020 -- Now}{Remote working}
Design and realization of the web console used by \textit{Irish banks} to manage the \textit{Pegasus payment system}.
\newline
\begin{itemize}
	\item Creation and documentation of the \textit{Pegasus} app architecture;
	\item Development of some pages, using the \textbf{Reactive Pattern};
	\item Training a colleague on the usage of \textbf{RxJS} and \textbf{Akita State Manager};
	\item Deployment of \textit{Pegasus} console on internal server and on \textbf{Netlify} using a custom \textbf{GitLab pipeline} and \textbf{Docker Runner};
	\item Creation of a status notification channel on \textbf{Microsoft Teams};
	\item Realization of a \textbf{Cypress} test suite executed at every code \textit{push};
	\item Documentation of the web console \textit{APIs} using \textbf{OpenAPI 3.0 specification};
	\item Automatic \textbf{TypeScript} models generation using \textbf{Swagger codegen};
	\item Active participation in internal strategic meetings and \textbf{working hand in hand with the Delivery Manager}.
\end{itemize}
\divider

\cvevent{Lead Front End Engineer}{\href{https://www.compass.it/}{GFT Italia S.r.l - for Compass}}{February 2020 -- August 2020}{Remote working}
\begin{itemize}
	\item Development of a \textbf{StencilJS WebComponent} fully customizable using \textbf{Google Analytics};
	\item Active participation in strategic meetings and \textbf{work hand in hand with the customer and its suppliers}.
\end{itemize}

\divider

\cvevent{Front End Developer}
{\href{https://www.compass.it/}{GFT Italia S.r.l - for Compass}}
{November 2019 -- March 2021}{Milan, Italy}
\begin{itemize}
	\item Maintenance and evolution of a \textbf{JQuery toolkit};
	\item Inclusion of \textbf{TensorFlow JS} for image recognition,
	\item Active participation in strategic meetings and \textbf{work hand in hand with the customer and its suppliers}.
\end{itemize}

\divider

\cvevent{Mobile Developer}
{\href{https://www.pagopa.gov.it/}{GFT Italia S.r.l - for PagoPA}}
{August 2019 -- February 2020}{Milan, Italy}
Maintenance and evolution of a mobile \textbf{SDK} for the payments to the \textit{italian public administration}.
\newline
\begin{adjustwidth}{0cm}{-8cm}
\begin{itemize}
	\item The \textit{Android} part of the SDK has been made in \textbf{Java};
	\item The \textit{iOS} part of the SDK has been made in \textbf{Objective C};
	\item The core \textit{business logic} has been made using the \textbf{native C++ code};
	\item Active participation in internal strategic meetings and \textbf{work hand in hand with the Delivery Manager}.
\end{itemize}

\divider

\cvevent{Code reviewer}
{\href{https://www.intesasanpaolo.com/}{GFT Italia S.r.l - for Intesa San Paolo}}
{October 2019 -- December 2019}{Milan, Italy}
\begin{itemize}
	\item Review, discussion and merge of the \textbf{Java} code produced by collegues.
	\item Training colleagues on \textbf{Git Flow}.
	\item Creation of a status notification channel on \textbf{Microsoft Teams}.
	\item \textit{Rest APIs} test automation using \textbf{Postman}.
\end{itemize}

\divider

\cvevent{Lead Full Stack Developer}
{\href{https://www.forexchange.it/}{GFT Italia S.r.l - for Maccorp Italiana S.p.A.}}
{June 2019 -- November 2020}{Milan, Italy}
Evolution of the new \textit{Forexchange.it} e-commerce.
\newline 
\begin{itemize}
	\item Migration of the \textbf{Angular} app from \textit{v5} to \textit{v6};
	\item Maintenance and evolution of the \textbf{NodeJS} \textit{Rest APIs};
	\item Insertion of a \textbf{Redis cache} to interact with an external tour operator;
	\item Setup of \textbf{GitLab pipelines} and \textbf{Redmine};
	\item Creation of a status notification channel on \textbf{Microsoft Teams};
	\item Migration of the payment system from \textit{Nexi} to \textit{Mercury - MonetaWeb};
	\item Active participation in strategic meetings and \textbf{work hand in hand with the Project Sponsor and the customer}.
\end{itemize}

\divider

\cvevent{Full Stack Developer}
{\href{https://www.chebanca.it/}{GFT Italia S.r.l - for CheBanca!}}
{September 2016 -- March 2021}{Milan, Italy}
Realization and Evolution of the \textit{CheBanca!'s 3.0 Multichannel Portal}.
\newline
The maximum team composition was \textbf{35 people}.
\newline
\begin{itemize}
	\item Realization of the front-end functionalities with \textbf{AngularJS}, all organized in \textbf{AMD modules} loaded by \textbf{RequireJS};
	\item Migration of the \textit{Rest APIs} from \textbf{Layer 7 API Management} to a \textbf{Spring Integration microservices} architecture;
	\item \textit{Microservices} tests with \textbf{Mockito} and \textbf{JUnit 4};
	\item Review and merge of all the code produced by me and my collegues;
	\item \textit{Rest APIs} documentation using \textbf{OpenAPI 2.0 specification};
	\item Maintenance and evolution of the \textit{UI Toolkit} used by all the portals;
	\item Creation of a \textit{PoC} to upgrade the \textit{UI Toolkit} to \textbf{WebComponents}, generated by \textbf{Yeoman};
	\item Work closely to the collegues, to help them overcome every obstacle;
	\item Training colleagues on code quality and the \textit{CheBanca architecture};
	\item Active participation in internal strategic meetings and \textbf{work hand in hand with the Delivery Manager and the customer}.
\end{itemize}

\divider

\cveventlocation{Company initiatives}
{\href{https://www.gft.com/it/it/index/}{GFT Italia S.r.l}}
{Milan, Italy}
\begin{itemize}
	\item Presentation about the \textbf{Clean Coding} importance;
	\item Migration of a \textit{microservices PoC} from company's server to \textbf{IBM Cloud} using \textbf{Kubernetes};
	\item \textit{Smart Classifier} project for the \textbf{IBM Watson Build 2018} competition;
	\item Presentation about the \textbf{Contract First approach to the API development};
	\item Technical recruiter.
\end{itemize}

\cvsection{Publications}

\nocite{*}

\printbibliography[heading=pubtype,
title={\printinfo{\faBook}{\href{https://ilteoood.medium.com/}{Medium articles}}},
type=book]

\end{adjustwidth}

\end{document}

